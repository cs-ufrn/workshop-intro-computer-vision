\documentclass{beamer}

\usepackage[utf8]{inputenc}
\usepackage[brazilian]{babel}
\usepackage{graphicx}
\usepackage{subcaption}
\usepackage{listings}

\title{Visão Computacional}
\subtitle{Introdução ao tópico}
\author{Vitor Greati\inst{1} \and Vinícius Campos\inst{1}}
\institute[]
{
	\inst{1}%
	Universidade Federal do Rio Grande do Norte
}
\date{}
\subject{Computer Science}

% Table of contents at the beginning of each section
\AtBeginSection[]
{
  \begin{frame}
    \frametitle{Sumário}
    \tableofcontents[currentsection, currentsubsection]
  \end{frame}
}

% Table of contents at the beginning of each subsection
%\AtBeginSubsection[]
%{
%  \begin{frame}
%    \frametitle{Table of Contents}
%    \tableofcontents[currentsection,currentsubsection]
%  \end{frame}
%}

\begin{document}

\frame{\titlepage}

\section{O gap semântico}

    \begin{frame}{O que você percebe nestas imagem?}

        \begin{figure}
            \centering
            \begin{subfigure}[b]{0.5\textwidth}
                \centering
                \includegraphics[height=2.6cm]{img/gcarpeople.jpg}
                \label{fig:carpeople}
            \end{subfigure}~
            \begin{subfigure}[b]{0.5\textwidth}
                \centering
                \includegraphics[height=2.6cm]{img/ghandwriting.jpg}
                \label{fig:handwriting}
            \end{subfigure}

            \begin{subfigure}[b]{0.5\textwidth}
                \centering
                \includegraphics[height=2.6cm]{img/gcat.jpg}
                \label{fig:carpeople}
            \end{subfigure}~
            \begin{subfigure}[b]{0.5\textwidth}
                \centering
                \includegraphics[height=2.6cm]{img/giris.jpg}
                \label{fig:handwriting}
            \end{subfigure}
        \end{figure}

        \pause

        A facilidade com que respondemos a essa pergunta
        se deve ao nosso sistema visual \textbf{nativo} 
        extremamente
        poderoso!

    \end{frame}

    \begin{frame}[fragile]{O que o computador percebe nessas imagens}{As matrizes de \emph{pixels}}

        À primeira vista\ldots

        \begin{columns}
            \begin{column}{0.4\textwidth}
        {\tiny
            %cars
        \begin{lstlisting}
[[ 42  23  19 ...,  21  29  25]
 [ 40  40  36 ...,  24  24  21]
 [ 28  30  36 ...,  30  13  27]
 ...,
 [115  78  45 ...,  28  36  17]
 [ 67  78 192 ...,  35  31  36]
 [ 67  79 104 ...,  34  32  31]]
        \end{lstlisting}}

        {\tiny
            % cat
        \begin{lstlisting}
[[138 137 137 ..., 107 107 107]
 [135 134 134 ..., 107 107 107]
 [130 129 129 ..., 107 107 107]
 ..., 
 [145 145 146 ..., 142 142 142]
 [146 145 144 ..., 144 144 145]
 [147 146 144 ..., 145 145 146]]
        \end{lstlisting}}
            \end{column}
            \begin{column}{0.5\textwidth}

        {\tiny
            % hand
        \begin{lstlisting}
[[222 224 224 ..., 204 201 200]
 [223 225 223 ..., 201 203 204]
 [226 226 226 ..., 204 202 205]
 ..., 
 [210 203 208 ..., 192 188 189]
 [206 206 207 ..., 190 188 189]
 [210 208 210 ..., 191 193 185]]
        \end{lstlisting}}

        {\tiny
            % iris
        \begin{lstlisting}
[[ 48  45  40 ...,  28  29  31]
 [ 45  46  43 ...,  28  29  30]
 [ 41  43  43 ...,  27  27  29]
 ..., 
 [101 101 103 ...,  64  51  32]
 [ 98  97  99 ...,  63  71  57]
 [ 97  97  97 ...,  38  57  65]]
        \end{lstlisting}}

            \end{column}
        \end{columns}

        \pause

        \begin{block}{Imagens digitais monocromáticas}
            Matrizes $I_j \in \mathbb{M}_{w_j \times h_j}([0,\ldots,255])$
 ou funções $f_j:\{1,\ldots,w_j\} \times \{1,\ldots,h_j\} 
 \to [0,255]$, onde $w_j$ é a largura e $h_j$ é a altura da imagem $j$.
        \end{block}


    \end{frame}
    
    \begin{frame}{O \emph{gap} semântico}{Percepção humana $\times$ Percepção da máquina}
    
    \end{frame}

\section{Visão Computacional}

    \subsection{Para além de \emph{pixels}}

    \begin{frame}{O \emph{gap} semântico}{Percepção humana vs. Percepção da máquina}
        Capturar fatos do mundo por uma ou mais imagens e realizar o inverso: reconstruir suas propriedades, 
        como formas, iluminação e distribuições de cor.
    \end{frame}

    \subsection{Aplicações}

    \subsection{Técnicas}

\section{Aprendizagem de Máquina}


\end{document}
